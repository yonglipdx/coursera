\chapter{编程技巧}

在判断两个浮点数a和b是否相等时,不要用\fn{a==b},应该判断二者之差的绝对值\fn{fabs(a-b)}是否小于某个阈值,例如\fn{1e-9}。

判断一个整数是否是为奇数,用\fn{x \% 2 != 0},不要用\fn{x \% 2 == 1},因为x可能是负数。

用\fn{char}的值作为数组下标(例如,统计字符串中每个字符出现的次数),要考虑到\fn{char}可能是负数。有的人考虑到了,先强制转型为\fn{unsigned int}再用作下标,这仍然是错的。正确的做法是,先强制转型为\fn{unsigned char},再用作下标。这涉及C++整型提升的规则,就不详述了。

以下是关于STL使用技巧的,很多条款来自《Effective STL》这本书。

\subsubsection{vector和string优先于动态分配的数组}

首先,在性能上,由于\fn{vector}能够保证连续内存,因此一旦分配了后,它的性能跟原始数组相当;

其次,如果用new,意味着你要确保后面进行了delete,一旦忘记了,就会出现BUG,且这样需要都写一行delete,代码不够短;

再次,声明多维数组的话,只能一个一个new,例如:
\begin{Code}
int** ary = new int*[row_num];
for(int i = 0; i < row_num; ++i)
    ary[i] = new int[col_num];
\end{Code}
用vector的话一行代码搞定,
\begin{Code}
vector<vector<int> > ary(row_num, vector<int>(col_num, 0));
\end{Code}

\subsubsection{使用reserve来避免不必要的重新分配}
